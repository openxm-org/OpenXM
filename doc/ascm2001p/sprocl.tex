%%UNIX --- UPDATED ON 13/8/97  
%====================================================================%
%                  sprocl.tex     27-Feb-1995                        %
% This latex file rewritten from various sources for use in the      %
% preparation of the standard proceedings Volume, latest version     %
% by Susan Hezlet with acknowledgments to Lukas Nellen.              %
% Some changes are due to David Cassel.                              %
%====================================================================%

\documentstyle[sprocl]{article}

\font\eightrm=cmr8

%\input{psfig}

\bibliographystyle{unsrt} %for BibTeX - sorted numerical labels by
                          %order of first citation.

\arraycolsep1.5pt

% A useful Journal macro
\def\Journal#1#2#3#4{{#1} {\bf #2}, #3 (#4)}

% Some useful journal names
\def\NCA{\em Nuovo Cimento}
\def\NIM{\em Nucl. Instrum. Methods}
\def\NIMA{{\em Nucl. Instrum. Methods} A}
\def\NPB{{\em Nucl. Phys.} B}
\def\PLB{{\em Phys. Lett.}  B}
\def\PRL{\em Phys. Rev. Lett.}
\def\PRD{{\em Phys. Rev.} D}
\def\ZPC{{\em Z. Phys.} C}

% Some other macros used in the sample text
\def\st{\scriptstyle}
\def\sst{\scriptscriptstyle}
\def\mco{\multicolumn}
\def\epp{\epsilon^{\prime}}
\def\vep{\varepsilon}
\def\ra{\rightarrow}
\def\ppg{\pi^+\pi^-\gamma}
\def\vp{{\bf p}}
\def\ko{K^0}
\def\kb{\bar{K^0}}
\def\al{\alpha}
\def\ab{\bar{\alpha}}
\def\be{\begin{equation}}
\def\ee{\end{equation}}
\def\bea{\begin{eqnarray}}
\def\eea{\end{eqnarray}}
\def\CPbar{\hbox{{\rm CP}\hskip-1.80em{/}}}%temp replacemt due to no font

%%%%%%%%%%%%%%%%%%%%%%%%%%%%%%%%%%%%%%%%%%%%%%%%%%%%%%%%%%%%%%%%%%%%%%%%
%%BEGINNING OF TEXT                           
%%%%%%%%%%%%%%%%%%%%%%%%%%%%%%%%%%%%%%%%%%%%%%%%%%%%%%%%%%%%%%%%%%%%%%%%

\begin{document}

\title{INSTRUCTIONS FOR PRODUCING A CAMERA-READY MANUSCRIPT 
USING WORLD SCIENTIFIC PUBLISHING STYLE FILES}

\author{A. B. AUTHOR, C. D. AUTHOR}

\address{World Scientific Publishing Co, 1060 Main Street, 
River Edge,\\ NJ 07661, USA\\E-mail: wspc@wspc.com} 

\author{A. N. OTHER}

\address{Department of Physics, Theoretical Physics, 1 Keble Road,\\
Oxford OX1 3NP, England\\E-mail: other@tp.ox.uk}

%%%%%%%%%%%%%%%%%%%%%%%%%%%%%%%%%%%%%%%%%%%%%%%%%%%%%%%%%%%%%%
% You may repeat \author \address as often as necessary      %
%%%%%%%%%%%%%%%%%%%%%%%%%%%%%%%%%%%%%%%%%%%%%%%%%%%%%%%%%%%%%%

\maketitle\abstracts{ This is where the abstract should
be placed. It should consist of one paragraph and give a
concise summary of the material in the article below.
Replace the title, authors, and addresses within the
curly brackets with your own title, authors, and
addresses; please use capital letters for the title and
the authors. You may have as many authors and addresses
as you wish. It's preferable not to use footnotes in
the abstract or the title; the acknowledgments for
funding bodies etc. are placed in a separate section at
the end of the text.}

\section{Guidelines}
\subsection{Producing the Hard Copy}\label{subsec:prod}
The hard copy may be printed using the advice given in
the file {\em splread.1st}, which is repeated in this
section. You should have three files in
total.\footnote{You can obtain these files from our WWW
pages at:

\noindent
{\sf http://www.wspc.co.uk/wspc/Styles/procs/procsread1st.html}

\noindent   
{\sf http://www.wspc.com.sg/others/style\_files/procsread1st.html}
\quad or by ftp from

\noindent
{\sf ftp.wspc.com.sg, cd /pub/style\_files/proceedings/SPROC/.}}

\noindent 
{\em splread.1st} --- the preliminary guide.

\noindent 
{\em sprocl.sty} --- the style file that provides the 
higher level latex commands for the proceedings. Don't 
change these parameters.

\noindent 
{\em sprocl.tex} --- the main text. You can delete our  
sample text and replace it with your own contribution 
to the volume, however we recommend keeping an initial 
version of the file for reference. Strip off any mail 
headers and then latex the tex file.  The command for 
latexing is {\sf latex sprocl}, do this twice to sort 
out the cross-referencing.

If you wish to use some other form of word-processor, 
some guidelines are given in Sec.~\ref{subsec:wpp} 
below. These files will work with standard latex 
2.09. If there is an abbreviation defined in the new
definitions at the top of the file {\em sprocl.tex} that
conflicts with one of your own macros, then delete the
appropriate command and revert to longhand. Failing
that, please consult your local texpert to check for
other conflicting macros that may be unique to your
computer system. Page numbers are included at the
bottom of the page for your guidance. Do not worry about
the final pagination of the volume which will be done
after you submit the paper.

\subsection{Using Other Word-Processing Packages}\label{subsec:wpp}
If you want to use some other form of word-processor to
construct your output, and are using the final hard copy version
of these files as guidelines; then please follow the style given 
here for headings, table and figure captions, and the footnote 
and citation marks. For this size of volume, the final page 
dimensions will be 8.5 by 6 inches (21.5 by 15.25 cm) however
you should submit the copy on standard A4 paper. The text area,
which includes the page numbers should be 7 by 4.7 inches (17.75
by 12 cm). The text should be in 10pt roman for the title,
section heads and the body of the text, and 9pt for the authors'
names and addresses. Please use capitals for the title and
authors, bold face for the title and headings, and italics for
the subheadings. The abstract, footnotes, figure and table
captions should be in 8pt.

It's also important to reproduce the spacing of the text and headings
as shown here.Text should be slightly more than single-spaced; use a
baselineskip (which is the average distance from the base of one line
of text to the base of an adjacent line) of 13 pts and 10 pts for
footnotes. All headings should be separated from the text preceding
it by a baselineskip of about 26 pts and use a baselineskip of about
18 pts for the following text.

Paragraphs should have a first line indented by about 0.25in (6mm)
except where the paragraph is preceded by a heading and the abstract
should be indented on both sides by 0.25in (6mm) from the main body of
the text.

\subsection{Headings and Text and Equations}
Please preserve the style of the headings, text fonts and line
spacing to provide a uniform style for the proceedings volume.

Equations should be centered and numbered consecutively, as in
Eq.~(\ref{eq:murnf}), and the {\em eqnarray} environment may be used
to split equations into several lines, for example in
Eq.~(\ref{eq:sp}), or to align several equations. An alternative
method is given in Eq.~(\ref{eq:spa}) for long sets of equations where
only one referencing equation number is wanted.

In latex, it is simplest to give the equation a label, as in
Eq.~(\ref{eq:murnf}) where we have used {\em
$\backslash$label\{eq:murnf\}} to identify the equation.  You can then
use the reference {\em $\backslash$ref\{eq:murnf\}} when citing the
equation in the text which will avoid the need to manually renumber
equations due to later changes. (Look at the source file for some
examples of this.)  

The same method can be used for referring to sections and subsections.

\subsection{Tables}
The tables are designed to have a uniform style throughout the
proceedings volume. It doesn't matter how you choose to place the
inner lines of the table, but we would prefer the border lines to be
of the style shown in Table~\ref{tab:exp}.  The top and bottom
horizontal lines should be single (using {\em $\backslash$hline}), and
there should be single vertical lines on the perimeter, (using {\em
$\backslash$begin\{tabular\}\{$|...|$\}}).  For the inner lines of the
table, it looks better if they are kept to a minimum. We've chosen a
more complicated example purely as an illustration of what is
possible.

The caption heading for a table should be placed at the top of 
the table.

\begin{table}[t]
\caption{Experimental Data bearing on 
$\Gamma(K \rightarrow \pi \pi \gamma)$
for the $K^0_S$, $K^0_L$ and $K^-$ mesons.\label{tab:exp}}
\vspace{0.2cm}
\begin{center}
\footnotesize
\begin{tabular}{|c|c|c|l|}
\hline
{} &\raisebox{0pt}[13pt][7pt]{$\Gamma(\pi^- \pi^0)\; s^{-1}$} &
\raisebox{0pt}[13pt][7pt]{$\Gamma(\pi^-\pi^0\gamma)\; s^{-1}$} &{}\\
\hline
\multicolumn{2}{|c|}{\raisebox{0pt}[12pt][6pt]{Process 
for Decay}} & &\\
\cline{1-2}
$K^-$   &$1.711 \times 10^7$ 
&\begin{minipage}{1in}
\begin{center}
$2.22 \times 10^4$ \\ (DE $ 1.46 \times 10^3)$
\end{center}
\end{minipage} 
&\begin{minipage}{1.5in}
\phantom{xxx}
No (IB)-E1 interference seen but data shows excess events 
relative to IB over
$E^{\ast}_{\gamma} = 80$ to $100$~MeV
\end{minipage} \\[22pt]
\hline
\end{tabular}
\end{center}
\end{table}

\subsection{Figures}
If you wish to `embed' a postscript figure in the file,
then remove the \% mark from the declaration of the
postscript figure within the figure description and
change the filename to an appropriate one. Also remove
the comment mark from the {\em input psfig} command at
the top of the file. You may need to play around with
this as different computer systems appear to use
different commands.

Next adjust the scaling of the figure until it's
correctly positioned, and remove the declarations of the
lines and any anomalous spacing. 

If you prefer to use some other method then it's very 
important to leave the correct amount of vertical space in 
the figure declaration to accomodate your figure 
(remove the lines and change the vspace in the example.) 
Send the hard copy figures on separate pages with clear
instructions to match them to the correct space in the 
final hard copy text. Please ensure the final hard copy 
figure is correctly scaled to fit the space available (this 
ensures the figure is legible.)

The caption heading for a figure should be placed below
the figure.

\subsection{Limitations on the Placement of Tables, 
Equations and Figures}\label{sec:plac} 
Very large figures and tables should be placed on a page by
themselves. One can use the instruction {\em
$\backslash$begin\{figure\}$[$p$]$} or {\em
$\backslash$begin\{table\}$[$p$]$} to position these,
and they will appear on a separate page devoted to
figures and tables. We would recommend making any
necessary adjustments to the layout of the figures and
tables only in the final draft. It is also simplest to
sort out line and page breaks in the last stages.

\subsection{Acknowledgments, Appendices, Footnotes and the Bibliography}
If you wish to have acknowledgments to funding bodies etc., 
these may be placed in a separate section at the end of the
text, before the Appendices. This should not be numbered so use
{\em $\backslash$section$\ast$\{Acknowledgments\}}.

It's preferable to have no appendices in a brief
article, but if more than one is necessary then simply
copy the {\em $\backslash$section$\ast$\{Appendix\}}
heading and type in Appendix A, Appendix B etc. between
the brackets.

Footnotes are denoted by a letter superscript in the
text,\footnote{Just like this one.} and references are
denoted by a number superscript.  We have used {\em
$\backslash$bibitem} to produce the bibliography.
Citations in the text use the labels defined in the
bibitem declaration, for example, the first paper by
Jarlskog~\cite{ja} is cited using the command {\em
$\backslash$cite\{ja\}}.

If you more commonly use the method of square brackets
in the line of text for citation than the superscript
method, please note that you need to adjust the
punctuation so that the citation command appears after
the punctuation mark.

\subsection{Final Manuscript}
The final hard copy that you send must be absolutely
clean and unfolded.  It will be printed directly without
any further editing. Use a printer that has a good
resolution (300 dots per inch or higher). There should
not be any corrections made on the printed pages, nor
should adhesive tape cover any lettering. Photocopies
are not acceptable.

The manuscript will not be reduced or enlarged when
filmed so please ensure that indices and other small
pieces of text are legible.

\section{Sample Text}
The following may be (and has been) described as
`dangerously irrelevant' physics. The Lorentz-invariant
phase space integral for a general n-body decay from a
particle with momentum $P$ and mass $M$ is given by:
\begin{equation}
I((P - k_i)^2, m^2_i, M) = \frac{1}{(2 \pi)^5}\!
\int\!\frac{d^3 k_i}{2 \omega_i} \! \delta^4(P - k_i).
\label{eq:murnf}
\end{equation}
The only experiment on $K^{\pm} \ra \pi^{\pm} \pi^0
\gamma$ since 1976 is that of Bolotov {\it et  
al}.~\cite{bu} There are two necessary conditions
required for any acceptable parametrization of the quark 
mixing matrix. The first is that the matrix must be
unitary, and the second is that it should contain a CP
violating phase $\delta$. In Sec.~\ref{subsec:wpp} the
connection between invariants (of form similar to J) and
unitarity relations will be examined further for the
more general $ n \times n $ case. The reason is that
such a matrix is not a faithful representation of the
group, i.e.~it does not cover all of the parameter space
available.
\begin{equation}
\begin{array}{rcl}
\bf{K} & = &  Im[V_{j, \alpha} {V_{j,\alpha + 1}}^*
{V_{j + 1,\alpha}}^* V_{j + 1, \alpha + 1} ] \\
&&{}+ Im[V_{k, \alpha + 2} {V_{k,\alpha + 3}}^*
{V_{k + 1,\alpha + 2}}^* V_{k + 1, \alpha + 3} ]  \\
&&{}+ Im[V_{j + 2, \beta} {V_{j + 2,\beta + 1}}^*
{V_{j + 3,\beta}}^* V_{j + 3, \beta + 1} ]  \\
&&{}+ Im[V_{k + 2, \beta + 2} {V_{k + 2,\beta + 3}}^*
{V_{k + 3,\beta + 2}}^* V_{k + 3, \beta + 3}] \\
& & \\
\bf{M} & = &  Im[{V_{j, \alpha}}^* V_{j,\alpha + 1}
V_{j + 1,\alpha} {V_{j + 1, \alpha + 1}}^* ]  \\
&&{}+ Im[V_{k, \alpha + 2} {V_{k,\alpha + 3}}^*
{V_{k + 1,\alpha + 2}}^* V_{k + 1, \alpha + 3} ]  \\
&&{}+ Im[{V_{j + 2, \beta}}^* V_{j + 2,\beta + 1}
V_{j + 3,\beta} {V_{j + 3, \beta + 1}}^* ]  \\
&&{}+ Im[V_{k + 2, \beta + 2} {V_{k + 2,\beta + 3}}^*
{V_{k + 3,\beta + 2}}^* V_{k + 3, \beta + 3}],
\\ & &
\end{array}\label{eq:spa}
\end{equation}
where $ k = j$ or $j+1$ and $\beta = \alpha$ or
$\alpha+1$, but if $k = j + 1$, then $\beta \neq \alpha
+ 1$ and similarly, if $\beta = \alpha + 1$ then $ k
\neq j + 1$.\footnote{An example of a matrix which has
elements containing the phase variable $e^{i \delta}$ to
second order, i.e.~elements with a phase variable
$e^{2i\delta}$ is given at the end of this section.}
There are only 162 quark mixing matrices using these
parameters which are to first order in the phase
variable $e^{i \delta}$ as is the case for the Jarlskog
parametrizations, and for which J is not identically
zero. It should be noted that these are physically
identical and form just one true parametrization.
\eject

\noindent
\bea
T & = & Im[V_{11} {V_{12}}^* {V_{21}}^* V_{22}]  \nonumber \\
&&{}+ Im[V_{12} {V_{13}}^* {V_{22}}^* V_{23}]   \nonumber \\
&&{}- Im[V_{33} {V_{31}}^* {V_{13}}^* V_{11}].
\label{eq:sp}
\eea

\begin{figure}[t]
\rule{5cm}{0.2mm}\hfill\rule{5cm}{0.2mm}
\vskip 2.5cm
\rule{5cm}{0.2mm}\hfill\rule{5cm}{0.2mm}
%\psfig{figure=filename.ps,height=1.5in}
\caption{A generalized cactus tree: the confluent
transfer-matrix $S$ transforms the state function $f(x)$ and
$f(z)$ into $f(x)$.  \label{fig:radish}}
\end{figure}

\section*{Acknowledgments}
This is where one places acknowledgments for funding
bodies etc.  Note that there are no section numbers for
the Acknowledgments, Appendix or References.

\section*{Appendix}
We can insert an appendix here and place equations so that they 
are given numbers such as Eq.~(\ref{eq:app}).
\be
x = y.
\label{eq:app}
\ee

\section*{References}
\begin{thebibliography}{99}
\bibitem{ja}C Jarlskog in {\em CP Violation}, ed. C Jarlskog
(World Scientific, Singapore, 1988).

\bibitem{ma}L. Maiani, \Journal{\PLB}{62}{183}{1976}.

\bibitem{bu}J.D. Bjorken and I. Dunietz, \Journal{\PRD}{36}{2109}{1987}.

\bibitem{bd}C.D. Buchanan {\it et al}, \Journal{\PRD}{45}{4088}{1992}.

\end{thebibliography}

\end{document}

%%%%%%%%%%%%%%%%%%%%%%%%%%%%%%%%%%%%%%%%%%%%%%%%%%%%%%%%%%%%%%%%%%%%%%%%%%
%% End of sprocl.tex  
%%%%%%%%%%%%%%%%%%%%%%%%%%%%%%%%%%%%%%%%%%%%%%%%%%%%%%%%%%%%%%%%%%%%%%%%%%




