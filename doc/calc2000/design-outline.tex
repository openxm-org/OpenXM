% $OpenXM: OpenXM/doc/calc2000/design-outline.tex,v 1.5 2000/04/26 08:18:30 noro Exp $

\section{Integration of Mathematical Software} 

As Schefstr\"om clarified in \cite{schefstrom},
integration of software tools has three dimensions:
data, control, and user interface.

Data integration concerns with the exchange of data between different
programs or the same program.
OpenMath \cite{OpenMath} and MP (Multi Protocol) \cite{GKW} are,
for example, general purpose mathematical data protocols.
They provide standard ways to express mathematical objects.
For example,
\begin{verbatim}
 <OMOBJ>  <OMI> 123 </OMI> </OMOBJ>
\end{verbatim}
means the (OpenMath) integer $123$ in OpenMath/XML expression.

Control integration concerns with the establishment and management of
inter-software communications.
Control involves, for example, a way to ask computations to other processes
and a method to interrupt computations on servers from a client.
RPC, HTTP, MPI, PVM are regarded as a general purpose control protocols or
infrastructures.
MCP (Mathematical Communication Protocol)
by Wang \cite{iamc} is such a protocol for mathematics.

Although data and control are orthogonal to each other, real world
requires both. The best way to evaluate and to improve such
integration schemes is to implement and to use them on various
platforms.  Dalmas et al. \cite{omimp} shows an implementation of
OpenMath API, where Maple, REDUCE and
AXIOM/Aldor are wrapped as servers.  MP$+$MCP \cite{iamc} shows a design
of server interface suited for interactive use and its limited
implementation on MAXIMA is reported.  Lakshman et al. \cite{pseware}
proposes functionalities which a server should have and Maple is
encapsulated as a server. These are all attempts to justify their
designs of protocols or architectures, but little is shown about their
practical usefulness, especially for developing real applications of
distributed computation.

In this paper we propose a unified server interface fitting for both
interactive use and efficient batch processing. 
It includes various supports for developing programs on distributed 
environment. We will show its
usability by implementing and using it on various platforms.

%NetSolve \cite{netsolve}, OpenMath$+$MCP, MP$+$MCP \cite{iamc},
%and MathLink \cite{mathlink} provide both data and control integration.
%Each integration method has their own features determined by their
%own design goals.

\section{Design Outline of OpenXM}

OpenXM (Open message eXchange protocol for Mathematics)
is a project aiming to integrate data, control and user interfaces
with design goals motivated by the following.

\begin{enumerate}
\item 
An interface for interactive distributed computations was introduced
to Risa/Asir \cite{asir} in 1995.
As an application,
a parallel speed-up was achieved for a Gr\"obner basis computation
(Noro and McKay \cite{noro-mckay}).
However, the protocol was local in Asir and we thought that we should
design an open protocol.
\item 
In order to implement algorithms in $D$-modules due to Oaku 
(see, e.g., \cite{sst-book}),
Kan/sm1 \cite{kan} called Risa/Asir as a UNIX external program to 
compute factorizations and primary ideal decompositions,
but the interface code was not clean.
We thought that it is necessary to provide a data and control protocol
for Risa/Asir to work as a server.

\item 
There are increasing number of ``expert'' systems which are useful
but which have their own interfaces and data formats.
A unified system integrating such systems
will be more convenient
for users who want to explore a new area of mathematics with these
software tools or users who need these systems only occasionally.

\item  We believe that an open integrated system is a future of mathematical
software.
However, it might be just a dream without realizability.
We want to build a prototype of such an open system by using
existing standards, technologies and several mathematical software tools.
We want to see how far we can go with this approach.
\end{enumerate}

Motivated with these, we started the OpenXM project with the following
fundamental architecture.
\begin{enumerate}
\item Communication is an exchange of messages. The messages are classified into
three types:
DATA, COMMAND, and SPECIAL.
They are called OX (OpenXM) messages.
Among the three types,
{\it OX data messages} wrap mathematical data.
We use standards of mathematical data formats such as OpenMath and MP
as well as our own data format {\it CMO}
({\it Common Mathematical Object format}).
\item Servers, which provide services to other processes, are stack machines.
The stack machine is called the
{\it OX stack machine}.
Existing mathematical software tools are wrapped with this stack machine.
Minimal requirements for a target wrapped with the OX stack machine
are as follows:
\begin{enumerate}
\item The target must have a serialized interface such as a character based
interface.
\item An output of the target must be understandable for computer programs;
it should follow a grammar that can be parsed with other software tools.
\end{enumerate}
\item Any server may have a hybrid interface;
it may accept and execute not only stack machine commands, 
but also its original command sequences.
For example,
if we send the following string to the {\tt ox\_asir} server 
(OpenXM server based on Risa/Asir \\
\verb+        " fctr(x^100-y^100); "      + \\
and call the stack machine command  \\
\verb+        SM_executeStringByLocalParser    + \\ 
then the server executes the asir command \\
\verb+ fctr(x^100-y^100); + 
(factorize $x^{100}-y^{100}$ over ${\bf Q}$)
and pushes the result onto the stack.
\item Network transparent supports for controlling servers are provided.
For example OpenXM defines a robust reset procedure to restart computations
without any confusion in I/O buffers.
It is very useful for debugging programs running on distributed environment.
\end{enumerate}
OpenXM package  is implemented on above fundamental architecture.
Currently the following servers are available in the OpenXM package
\cite{openxm-web}.

\begin{description}
\item{\tt ox\_asir}
A server for Risa/Asir, a general-purpose computer algebra
system. It provides almost
all functionalities of Risa/Asir such as polynomial factorization,
Gr\"obner basis computation and primary ideal decomposition.
\item{\tt ox\_sm1}
A server for Kan/sm1, a system for computation in 
the ring of differential operators including computation of Gr\"obner bases
and cohomology groups.
\item {\tt ox\_phc}
A server for PHC pack \cite{phc}, a general-purpose solver for
polynomial systems by homotopy continuation. 
\item {\tt ox\_tigers}
A server for TiGERS \cite{tigers}, a system to enumerate
all Gr\"obner bases of affine toric ideals.
It can be used to determine the state polytope
of a given affine toric ideal.
\item {\tt ox\_gnuplot}
A server for GNUPLOT, a famous plotting tool.
\item {\tt ox\_math}
A server for Mathematica.
\item {\tt OMproxy}
A server for translation between CMO and OpenMath/XML expressions.
It is written in Java.
This module provides Java classes OXmessage, CMO, and SM
for the OpenXM protocol, too.
\end{description}
In addition to these servers, Risa/Asir, Kan/sm1 and Mathematica
can act as clients. 
For example, the following is a command sequence to ask $1+1$ from
the Asir client to the {\tt ox\_sm1} server:
\begin{verbatim}
  P = sm1_start();
  ox_push_cmo(P,1); ox_push_cmo(P,1);
  ox_execute_string(P,"add"); ox_pop_cmo(P);
\end{verbatim}
The OpenXM package is implemented on the  OpenXM for TCP/IP, 
which uses the client-server model.
The OpenXM on MPI \cite{MPI} is currently running on Risa/Asir
as we will see in Section \ref{section:homog}.
In this paper, we discuss only on systems for TCP/IP
to concentrate on the core part of our design.
Note that a C library interface is available for some servers.





