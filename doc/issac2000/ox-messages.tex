%%$OpenXM: OpenXM/doc/issac2000/ox-messages.tex,v 1.1 2000/01/02 07:32:12 takayama Exp $

\section{OX messages}  (Ohara)

An OX message for TCP/IP is a byte stream consisting of
a header and a body.
\begin{center}
\begin{tabular}{|c|c|}
\hline
Header	& \hspace{10mm} Body \hspace{10mm} \\
\hline
\end{tabular}
\end{center}
The header consists of two signed 32 bit integers.
The first one is an OX tag 
and the second one is a serial number of the OX message.
Negative numbers are expressed by the two's complement.
Several byte orders including the network byte order
are allowed and the byte order is determined as a part of
the establishment of a connection.

The OX messages are classifed into three types:
DATA, COMMAND, and others.
We have the following main tags for the OX messages.
\begin{verbatim}
#define	OX_COMMAND               513  // COMMAND
#define	OX_DATA	                 514  // DATA
#define OX_SYNC_BALL             515  // others
#define OX_DATA_WITH_LENGTH      521  // DATA
#define OX_DATA_OPENMATH_XML     523  // DATA
#define OX_DATA_OPENMATH_BINARY  524  // DATA
#define OX_DATA_MP               525  // DATA
\end{verbatim}

New OX tags may be added.
The new tag should be classified into DATA or COMMAND.
For example, \verb+ OX_DATA_ASIR_LOCAL_BINARY +  was added a few month ago
to send internal serialized objects of asir via the OpenXM protocol.
This is a tag classifed to DATA.
See the home page of OpenXM to add a new tag.

In OpenXM, a distributed computation is done as follows:
\begin{enumerate}
\item A client requests something to a server.
\item The server does works according to the request.
\item The client requests to send data to the server.
\item The server sends the data to the client and the client gets the data.
\end{enumerate}
The server is a stackmachine. 
That is {\it OX data} message sent by the client
are pushed to the stack of the server. 
If the server gets an {\it OX command} message, then the data are
popped form the stack and they are used as arguments of a function call.
For example, the asir command
\verb+ ox_push_cmo(P,1) + (push integer $1$ to the server P)
sends an OX data message
(OX\_DATA,(CMO\_ZZ,1)) to the server $P$.
Here,
OX\_DATA stands for OX\_DATA header and 
(CMO\_ZZ,1) is a body standing for $1$ expressed 
in the CMO data encoding format.





