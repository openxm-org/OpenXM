%$OpenXM: OpenXM/doc/issac2000/issac2000.tex,v 1.4 2000/01/07 06:27:55 noro Exp $
%% You need acmconf.cls and flushend.sty to compile this file.
%% They may be obtained from 
%%  http://riksun.riken.go.jp/archives/tex-archive/macros/latex/contrib/supported/acmconf/
\documentclass[submit]{acmconf}
%\documentclass{article}
%% \CopyrightText{\copyright 2000, }
\IfFileExists{graphicx.sty}{\usepackage{graphicx}}{}
\IfFileExists{epsfig.sty}{\usepackage{epsfig}}{}
\ConferenceName{1. ISSAC 2000, St. Andrews, UK, 2000}
\ConferenceShortName{ISSAC2000}
\def\OpenXM{{\rm OpenXM\ }}

\begin{document}
\date{January 16, 2000}
\title{OpenXM 
      --- an Open System \\ to Integrate Mathematical Softwares}
\author{\Author{Masahide Maekawa}\\
         \Address{Kobe University}\\
         \Email{maekawa@math.kobe-u.ac.jp}\\
         \and
         \Author{Masayuki Noro}\\
         \Address{Fujitsu Labs}\\
         \Email{noryo@flab.fujitsu.co.jp}
         \and
         \Author{Katsuyoshi Ohara}\\
         \Address{Kanazawa University}\\
         \Email{ohara@air.s.kanazawa-u.ac.jp}
         \and
         \Author{Yukio Okutani}\\
         \Address{Kobe University}\\
         \Email{okutani@math.kobe-u.ac.jp}
         \and
         \Author{Nobuki Takayama}\\
         \Address{Kobe University}\\
         \Email{takayama@math.kobe-u.ac.jp}
         \and
         \Author{Yasushi Tamura}\\
         \Address{Kobe University}\\
         \Email{tamura@math.kobe-u.ac.jp}
       }
\maketitle

\begin{abstract}
\OpenXM is a free, or Open Source, infrastructure for mathematical
softwares.
It provides methods and protocols 
for interactive distributed computation and
to integrate many kind of mathematical softwares.
\OpenXM package is a set of softwares that supports \OpenXM protocols.
It is currently a collection of softwares
{\tt Risa/Asir} \cite{asir}, {\tt Kan/sm1} \cite{kan}, {\tt PHC} pack \cite{phc}, {\tt GNUPLOT},
{\tt Mathematica} interface, and
{\tt OpenMath}/XML \cite{OpenMath} interpreter.
These are wrapped with the \OpenXM stackmachine
to connect each other.
{\it Availability}: The OpenXM package is obtainable from \cite{openxm-web}
from January 24, 2000.
\end{abstract}

\begin{keywords}
Asir,
IAMC, Interactive Distributed Computation, 
Integration of Mathematical Softwares,
MP, OpenMath, OpenXM. 
\end{keywords}

%\section{Introduction}

% $OpenXM: OpenXM/doc/issac2000/design-outline.tex,v 1.5 2000/01/11 05:35:48 noro Exp $

\section{Design Outline} 

As Schefstr\"om clarified in \cite{schefstrom},
integration of tools and softwares has three dimensions:
data, control, and user interface.

Data integration concerns with the exchange of data between different
softwares or same softwares.
OpenMath \cite{OpenMath} and MP (Multi Protocol) \cite{GKW} are,
for example, general purpose mathematical data protocols.
They provide standard ways to express mathematical objects.
For example,
\begin{verbatim}
 <OMOBJ>  <OMI> 123 </OMI> </OMOBJ>
\end{verbatim}
means the (OpenMath) integer $123$ in OpenMath/XML expression.

Control integration concerns with the establishment and management of
inter-software communications.
Control involves, for example, a way to ask computations to other processes
and a method to interrupt computations on servers from a client.
RPC, HTTP, MPI, PVM are regarded as a general purpose control protocols or
infrastructures.
MCP (Mathematical Communication Protocol)
by Wang \cite{iamc} is such a protocol specialized to mathematics.

Although, data and control are orthogonal to each other,
real world requires both.
NetSolve \cite{netsolve}, OpenMath$+$MCP, MP$+$MCP \cite{iamc},
and MathLink \cite{mathlink} provide both data and control integration.
Each integration method has their own special features due to their
own design goals and design motivations.
OpenXM (Open message eXchange protocol for Mathematics)
is a project aiming to integrate data, control and user interfaces
with itw own set of design goals.
To explain our design outline, we start with a list of
our motivations.
\begin{enumerate}
\item Noro has developed a general
purpose computer algebra system Risa/Asir \cite{asir}.
A set of functions for interactive distributed computations were introduced
in Risa/Asir version 950831 released in 1995.
The model of computation was RPC (remote procedure call)
and it had its own serialization method for objects.
A robust interruption method was provided by having two communication channels
like ftp.
As an application of this robust and interactive distributed computation
system, speed-up was achieved for a huge Gr\"obner basis computation
to determine all odd order replicable functions 
by Noro and McKay \cite{noro-mckay}.
However, the protocol was closed in Asir and we thought that we should
design an open protocol.
\item Takayama has developed
a special purpose computer algebra system Kan/sm1 \cite{kan},
which is a Gr\"obner engine for the ring of differential operators $D$ and
a package for computational algebraic geometry via D-module computations.
In order to implement algorithms in D-modules due to Oaku 
(see, e.g., \cite{sst-book}),
factorizations and primary ideal decompositions were necessary.
Kan/sm1 does not have an implementation for these and called
Risa/Asir as a C library or a UNIX external program.
This approach was not satisfactory.
Especially, we could not write a clean interface code between these
two systems.
We thought that it is necessary to provide a data and control protocol
for Risa/Asir to work as a server of factorization and primary ideal
decomposition.
\item The number of mathematical softwares is increasing rapidly in the last
decade of the 20th century.
These are usually ``expert'' systems for one area of mathematics
such as ideals, groups, numbers, polytopes, and so on.
They have their own interfaces and data formats.
Interfaces are usually specialized to a specific field of mathematics
or poor because developers do not have time for designing user interface
languages.
It is fine for intensive and serious users of these systems.
%% x2 stands for x^2, specialized for polynomial ring.
However, for users who want to explore a new area of mathematics with these
softwares or users who need these systems only occasionally,
a unified system will be more convenient.
For example, if we can call and use mathematical softwares
like CoCoa, GAP, Macaulay2, Porta, Singular, Snapea, $\ldots$
from Aldor, Asir, Axiom, Maple, Magma, muPAD, Mathematica, and so on,
it will be wonderful in research and education
of mathematics. This is an unification of user interfaces of mathematical
softwares.
\item  We believe that an open integrated system is a future of mathematical
softwares.
However, it might be just a dream without realizability.
We want to build a prototype system of such an open system by using
existing standards, technologies and several mathematical softwares.
We want to see how far we can go with this approach.
\end{enumerate}

Motivated with these, we started the OpenXM project with the following
fundamental architecture.
\begin{enumerate}
\item Communication is an exchange of messages. The messages are classified into
three types:
DATA, COMMAND, and others.
The messages are called OX (OpenXM) messages.
Mathematical data are wrapped with {\it OX messages}.
We use standards of mathematical data formats such as OpenMath and MP
and our own data format ({\it CMO --- Common Mathematical Object format})
as data expressions.
\item Servers, which provide services to other processes, are stack machines.
The stack machine is called the
{\it OX stack machine}.
Existing mathematical softwares are wrapped with this stack machine.
Minimal requirements for a target software wrapped with the OX stack machine
are as follows:
\begin{enumerate}
\item The target must have a serialized interface such as a character based
interface.
\item An output of the target must be understandable for computer programs;
it should follow a grammar that can be parsed with other softwares.
\end{enumerate}
\end{enumerate}
We are implementing a package, OpenXM package,  
which aims to realize our wishes stated as motivations.
It is based on above fundamental architecture.
For example, the following is a command sequence to ask $1+1$ from
the Asir client to the OX sm1 server:
\begin{verbatim}
  P = sm1_start();
  ox_push_cmo(P,1); ox_push_cmo(P,1);
  ox_execute_string(P,"add"); ox_pop_cmo(P);
\end{verbatim}
The current system, OpenXM on TCP/IP, 
uses client-server model and the TCP/IP is used for interprocess
communications.
The OpenXM on MPI \cite{MPI} is currently running on Risa/Asir
as we will see Section \ref{section:homog}.
However, we focus only on the system based on TCP/IP in this paper.






%%$OpenXM: OpenXM/doc/issac2000/ox-messages.tex,v 1.3 2000/01/11 05:35:48 noro Exp $

\section{OX messages}

An OX message for TCP/IP is a byte stream consisting of
a header and a body.
\begin{center}
\begin{tabular}{|c|c|}
\hline
Header	& \hspace{10mm} Body \hspace{10mm} \\
\hline
\end{tabular}
\end{center}
The header consists of two signed 32 bit integers.
The first one is an OX tag 
and the second one is a serial number of the OX message.
Negative numbers are expressed by the two's complement.
Several byte orders including the network byte order
are allowed and the byte order is determined as a part of
the establishment of a connection. See Section \ref{secsession}.

The OX messages are classified into three types:
DATA, COMMAND, and others.
We have the following main tags for the OX messages.
\begin{verbatim}
#define	OX_COMMAND               513  // COMMAND
#define	OX_DATA	                 514  // DATA
#define OX_SYNC_BALL             515  // others
#define OX_DATA_WITH_LENGTH      521  // DATA
#define OX_DATA_OPENMATH_XML     523  // DATA
#define OX_DATA_OPENMATH_BINARY  524  // DATA
#define OX_DATA_MP               525  // DATA
\end{verbatim}

New OX tags may be added.
The new tag should be classified into DATA or COMMAND.
For example, \verb+ OX_DATA_ASIR_LOCAL_BINARY +  was added a few month ago
to send internal serialized objects of asir via the OpenXM protocol.
This is a tag classified to DATA.
See the web page of OpenXM to add a new tag.

In OpenXM, a distributed computation is done as follows:
\begin{enumerate}
\item A client requests something to a server.
\item The server does some work according to the request.
\item The client requests to send data to the server.
\item The server sends the data to the client and the client gets the data.
\end{enumerate}
The server is a stack machine. (see Section~\ref{sec:ox-stackmachines}
for detail)
That is {\it OX data} message sent by the client
are pushed to the stack of the server. 
If the server gets an {\it OX command} message, then the server extract
a stack machine code in the OX command message and interpret the code.
For example, in case of SM\_executeFunction, some data are popped from
the stack and they are used as arguments of a function call.

We explain an implementation of handling OX messages.
For example, the asir command {\tt ox\_push\_cmo(P,1)}
(push integer $1$ to the server $P$)
sends an OX data message
{\tt (OX\_DATA,(CMO\_ZZ,1))} to the server $P$.
Here,
OX\_DATA stands for OX\_DATA header and 
{\tt (CMO\_ZZ,1)} is a body standing for $1$ expressed 
in the CMO data encoding format.
The server tranlates $(CMO\_ZZ, 1)$ to its own internal object fotrmat
for integers and pushs the object to the stack.

An OpenXM client admit that its own command sends some OX messages
sequentially at once.  

For example, the asir command
{\tt ox\_execute\_string(P, "Print[x+y]")} sends an OX data message
{\tt (OX\_DATA, (CMO\_STRING, "Print[x+y]"))} and an OX command message
{\tt (OX\_COMMAND, (SM\_executeStringByLocalParser))} to an OpenXM
server.

% $OpenXM: OpenXM/doc/issac2000/data-format.tex,v 1.9 2000/01/16 10:55:40 takayama Exp $

\section{Data Format}   

OpenXM admits multiple mathematical encodings such as OpenMath, MP, CMO
(Common Mathematical Object format).
OpenXM itself does not exhibit a bias towards a particular encodings 
as a main mathematical data carrier and an OpenXM compliant system do not need to
implement all possible data formats.
However  they should at least implement seven primitive 
data types of the CMO, which are necessary to 
carry several control informations such as a {\it mathcap}.
Mathcap is a list of supported CMO's, OpenXM stack machine codes, 
and necessary extra informations.
If a program sends an OX messages unknown to its peer, 
an unrecoverable error may occur. 
By exchanging mathcaps, a program knows its peer's capability 
and such an error can be avoided.
Mathcap is also defined as a CMO.
See \cite{noro-takayama} for the details.

Encoding types of OX data are distinguished with tags
of OX messages.
For example,
an OX message with the tag 
{\tt OX\_DATA} is followed by a CMO packet.
An OX message with the tag 
{\tt OX\_DATA\_OPENMATH\_XML} is followed by 
an OpenMath XML string.

Let us explain the data format of CMO.
Any CMO packet consists of a header and a body.
The size of the header is 4 bytes that tags the data type of the body.
Data type tags are signed 32 bit integers which is called {\sl int32} in this
paper.
Following tags are registered in the OpenXM.
\begin{verbatim}
CMO_ERROR2, CMO_NULL, CMO_INT32, CMO_DATUM, CMO_STRING,
CMO_MATHCAP, CMO_LIST, CMO_MONOMIAL32, CMO_ZZ, CMO_QQ, CMO_ZERO,
CMO_DMS_GENERIC, CMO_DMS_OF_N_VARIABLES, CMO_RING_BY_NAME,
CMO_RECURSIVE_POLYNOMIAL, CMO_LIST_R, CMO_INT32COEFF,
CMO_DISTRIBUTED_POLYNOMIAL, CMO_POLYNOMIAL_IN_ONE_VARIABLE,
CMO_RATIONAL, CMO_64BIT_MACHINE_DOUBLE,
CMO_ARRAY_OF_64BIT_MACHINE_DOUBLE, CMO_BIGFLOAT,
CMO_IEEE_DOUBLE_FLOAT, CMO_INDETERMINATE, CMO_TREE, CMO_LAMBDA
\end{verbatim}
The first seven primitive types should be implemented 
on all OpenXM compliant systems.
The formats are as follows. \\
\begin{tabular}{|c|c|}
\hline
{\sl int32} {\tt CMO\_ERROR2} & {\sl CMObject} {\rm ob} \\ 
\hline
\end{tabular} \\
\begin{tabular}{|c|c|}
\hline
{\sl int32} {\tt CMO\_NULL}  \\ 
\hline
\end{tabular} \\
\begin{tabular}{|c|c|}
\hline
{\sl int32} {\tt CMO\_INT32}& {\sl int32} {\rm n}  \\ 
\hline
\end{tabular} \\
\begin{tabular}{|c|c|c|c|c|}
\hline
{\sl int32} {\tt CMO\_STRING}& {\sl int32} {\rm n} & {\sl byte} {\rm data[0]} $\cdots$ & {\sl byte} {\rm data[n-1]} \\  \hline
\end{tabular} \\
\begin{tabular}{|c|c|}
\hline
{\sl int32} {\tt CMO\_MATHCAP} & {\sl CMObject} {\rm ob} \\ 
\hline
\end{tabular} \\
\begin{tabular}{|c|c|c|c|c|}
\hline
{\sl int32} {\tt CMO\_LIST}& {\sl int32} {\rm n} & {\sl CMObject} {\rm ob[0]} 
$\cdots$ & {\sl CMObject} {\rm ob[n-1]} \\  \hline
\end{tabular} \\

As to the formats of other CMO's, see \cite{noro-takayama}.

When one wants to implement CMO on a server, the person proceeds 
as follows.
\begin{enumerate}
\item Look for the list of CMO's at the web cite \cite{openxm-web}.
If there is a CMO that fits to one's requirement, then use this CMO.     
\item If there is no suitable CMO, design a new CMO and register 
the new CMO to \cite{openxm-web} with a description and examples.
\end{enumerate}


% $OpenXM: OpenXM/doc/issac2000/openxm-stackmachines.tex,v 1.8 2000/01/15 03:46:27 noro Exp $

\section{OpenXM Stack machines}\label{sec:ox-stackmachines}

In OpenXM specification, all servers are stack machines.
%These are called OpenXM stack machines.
When a server ox\_xyz gets an OX data message,
it translates the data into a local object of ox\_xyz
and pushes the object onto the stack.
According to the OpenMath specification, 
the translation scheme together with definitions of mathematical operations
of the system ox\_xyz is called the {\it PhraseBook} of ox\_xyz.

Any OX command message starts with the int32 tag OX\_COMMAND.
The body is an OpenXM stack machine operation code expressed by int32.
The codes are listed below.
\begin{verbatim}
#define SM_popSerializedLocalObject               258
#define SM_popCMO                                 262
#define SM_popString                              263
#define SM_mathcap                                264
#define SM_pops                                   265
#define SM_setName                                266
#define SM_evalName                               267
#define SM_executeStringByLocalParser             268
#define SM_executeFunction                        269
#define SM_beginBlock                             270
#define SM_endBlock                               271
#define SM_shutdown                               272
#define SM_setMathCap                             273
#define SM_executeStringByLocalParserInBatchMode  274
#define SM_getsp                                  275
#define SM_dupErrors                              276
#define SM_control_kill                          1024
#define SM_control_to_debug_mode                 1025
#define SM_control_exit_debug_mode               1026
#define SM_control_reset_connection              1030
\end{verbatim}

OpenXM does not have a standard for mathematical operation sets
while it is a work in progress in the GAP group \cite{gap}.
Each OpenXM server has its own set of mathematical operations,
which are performed as follows.
First, arguments for a mathematical operation
and the number of the arguments are pushed.
Second, 
the mathematical operator name, 
such as {\tt fctr} (asir factorization command),
is pushed as a string.
Finally, the stack machine command
{\tt SM\_executeFunction} (269) evaluates the operator and
pushes the result onto the stack
after poping the operator name, the number of arguments
and arguments.
For example, the following code factorizes $x^{100}-1$ by calling
{\tt ox\_asir} from asir.
\begin{verbatim}
P = ox_launch(); 
ox_push_cmo(P,x^100-1); ox_push_cmo(P,ntoint32(1));
ox_push_cmd(P,269); 
Ans = ox_pop_cmo(P);
\end{verbatim}

When an error has occurred on an OpenXM server,
an error object is pushed to the stack instead of a result of the computation.
The error object consists of the serial number of the OX message
which caused the error, and an error message.
\begin{verbatim}
[341] ox_rpc(0,"fctr",1.2*x)$
[342] ox_pop_cmo(0);
error([8,fctr : invalid argument])
\end{verbatim}

OpenXM server won't send error messages to the client
except when it receives a {\tt SM\_pop*} command.
OX stackmachines work in the asynchronous mode which is similar 
to X servers.
For servers for graphic and sound applications, it is an advantageous feature.
It is also easy to emulate RPC and a web server for MCP \cite{iamc} 
on our asynchronous OX stackmachines.







% $OpenXM: OpenXM/doc/issac2000/session-management.tex,v 1.1 1999/12/23 10:25:09 takayama Exp $

\section{Session Management}  (Noryo)

MEMO: key words:
Security (ssh PAM), initial negotiation of byte order,
mathcap, interruption, debugging window, etc.
 

% $OpenXM: OpenXM/doc/issac2000/openxm-clients.tex,v 1.5 2000/01/15 06:26:06 takayama Exp $

\section{OpenXM Clients}    

\subsection{Risa/Asir}

Risa/Asir provides a launcher to invoke an OpenXM server and to set up the
communication between the server and itself. As a client, 
it provides many built-in functions for communication.

\subsubsection{Setting up servers}
{\tt ox\_launch} is a general purpose launcher.  This application
invokes a server and sets up the server-client communication
according to the protocol stated in Section \ref{launcher}, then
itself becomes a control server.
Several facilities related to {{\tt ox\_launch}} are provided
as built-in functions of Risa/Asir: a function to invoke a server
automatically from a give host name and a server name, and a set 
of functions to execute the port generation, {\tt bind}, {\tt listen},
{\tt connect} and {\tt accept} operations on sockets individually.

\subsubsection{Manipulating servers}
Fundamental operations on OpenXM servers are 
exchange of {\tt OX} data and sending of {\tt SM} commands.
The following functions
are provided to execute these primitive operations:
{\tt ox\_push\_cmo()} for pushing data to a server, 
{\tt ox\_push\_cmd()} for sending an {\tt SM} command to a server
and {\tt ox\_get()} for receiving data from a stream.

Some operations including the reset operation are realized by
combining these primitives.  Among them, frequently used ones are
provided as built-in functions. We show several ones.

\begin{itemize}
\item {\tt ox\_pop\_cmo()}

It requests a server to send data on the stack to the stream, then
it receives the data from the stream.

\item {\tt ox\_cmo\_rpc()}

After pushing the name of a function, arguments and the number of the
arguments to the stack of a server, it requests the server to execute
the function. It does not wait the termination of the function call.

\item {\tt ox\_reset()}

After sending {\tt SM\_control\_reset\_connection} to a control server,
it completes the operations stated in Section \ref{control}.
\end{itemize}
Furthermore {\tt ox\_select()} is provided to detect streams ready for
reading. It is realized by the {\tt select()} system call and is used
to avoid blocking on read operations.

\subsection{Mathematica}

We provide an OpenXM client {\tt math2ox} written as an external module
for Mathematica.  Our client communicates with Mathematica by MathLink and
with an OpenXM server by OpenXM protocols.  
By using the module {\tt math2ox},
we can call OpenXM servers from Mathematica;
here is an example of a computation of the de Rham cohomology groups
of ${\bf C}^2 \setminus V(x^3-y^2)$.
{\footnotesize
\begin{verbatim}
In[1]:= Install["math2ox"]
In[2]:= OxStart["../lib/sm1/bin/ox_sm1_forAsir"]
In[3]:= OxExecute[" [(x^3-y^2) (x,y)] deRham "]
In[4]:= OxPopString[]
Out[4]=  [ 1 , 1 , 0 ] 
\end{verbatim}
}

\subsubsection{Functions}

The {\tt math2ox} has the following functions.
\begin{quote}
{\tt OxStart[s\_String]} \\
{\tt OxStartInsecure[s\_String]} \\
{\tt OxExecuteString[s\_String]}  \\
{\tt OxParse[s\_String]} \\
{\tt OxGet[]} \\
{\tt OxPopCMO[]} \\
{\tt OxPopString[]} \\
{\tt OxClose[]} \\
{\tt OxReset[]}
\end{quote}
For example, {\tt OxPopCMO[]} executes the same operation
as {\tt ox\_pop\_cmo()} in Risa/Asir.
By using the {\tt OxParse[]} function, one can send suitable OX messages,
written by the OX expression, to a server. OX expressions are
Lisp-like expressions for OX messages and are defined
in~\cite{noro-takayama}.  
The {\tt OxGet[]} receives an OX data message
and returns its translation to an local object.


%$OpenXM$

\section{OpenXM CVS server}  

OpenXM package is currently developed and maintained on
a CVS server.
The CVS is an open system for a version control.
Any mathematical programmer is welcome to
join with the committer team of OpenXM.
The CVS web is used for reading the source tree and the change logs.
The cvsup server is used to synchronize the local source tree with
the main source tree.
These software tools are used to develop free software packages
like the FreeBSD operating system.
These tools are very useful in development
of the OpenXM mathematical software package by a team.


% $OpenXM: OpenXM/doc/issac2000/homogeneous-network.tex,v 1.8 2000/01/16 03:15:49 noro Exp $

\subsection{Distributed computation with homogeneous servers}
\label{section:homog}

One of the aims of OpenXM is a parallel speedup by a distributed computation
with homogeneous servers. As the current specification of OpenXM does
not include communication between servers, one cannot expect
the maximal parallel speedup. However it is possible to execute
several types of distributed computation as follows.

\subsubsection{Product of univariate polynomials}

Shoup \cite{Shoup} showed that the product of univariate polynomials
with large degrees and large coefficients can be computed efficiently
by FFT over small finite fields and Chinese remainder theorem.
It can be easily parallelized:

\begin{tabbing}
Input :\= $f_1, f_2 \in {\bf Z}[x]$ such that $deg(f_1), deg(f_2) < 2^M$\\
Output : $f = f_1f_2$ \\
$P \leftarrow$ \= $\{m_1,\cdots,m_N\}$ where $m_i$ is an odd prime, \\
\> $2^{M+1}|m_i-1$ and $m=\prod m_i $ is sufficiently large. \\
Separate $P$ into disjoint subsets $P_1, \cdots, P_L$.\\
for \= $j=1$ to $L$ $M_j \leftarrow \prod_{m_i\in P_j} m_i$\\
Compute $F_j$ such that $F_j \equiv f_1f_2 \bmod M_j$\\
\> and $F_j \equiv 0 \bmod m/M_j$ in parallel.\\
\> (The product is computed by FFT.)\\
return $\phi_m(\sum F_j)$\\
(For $a \in {\bf Z}$, $\phi_m(a) \in (-m/2,m/2)$ and $\phi_m(a)\equiv a \bmod m$)
\end{tabbing}

Figure \ref{speedup}
shows the speedup factor under the above distributed computation
on Risa/Asir. For each $n$, two polynomials of degree $n$
with 3000bit coefficients are generated and the product is computed.
The machine is Fujitsu AP3000,
a cluster of Sun connected with a high speed network and MPI over the
network is used to implement OpenXM.
\begin{figure}[htbp]
\epsfxsize=8.5cm
\epsffile{speedup.ps}
\caption{Speedup factor}
\label{speedup}
\end{figure}

If the number of servers is $L$ and the inputs are fixed, then the cost to
compute $F_j$ in parallel is $O(1/L)$, whereas the cost
to send and receive polynomials is $O(L)$ if {\tt ox\_push\_cmo()} and
{\tt ox\_pop\_cmo()} are repeatedly applied on the client.
Therefore the speedup is limited and the upper bound of
the speedup factor depends on the ratio of 
the computational cost and the communication cost for each unit operation.
Figure \ref{speedup} shows that 
the speedup is satisfactory if the degree is large and $L$
is not large, say, up to 10 under the above envionment.
If OpenXM provides the broadcast and the reduce operations, the cost of 
sending $f_1$, $f_2$ and gathering $F_j$ may be reduced to $O(log_2L)$
and we can expect better results in such a case.

\subsubsection{Competitive distributed computation by various strategies}

SINGULAR \cite{Singular} implements {\tt MP} interface for distributed
computation and a competitive Gr\"obner basis computation is
illustrated as an example of distributed computation.
Such a distributed computation is also possible on OpenXM.
The following Risa/Asir function computes a Gr\"obner basis by
starting the computations simultaneously from the homogenized input and
the input itself.  The client watches the streams by {\tt ox\_select()}
and the result which is returned first is taken. Then the remaining
server is reset.

\begin{verbatim}
/* G:set of polys; V:list of variables */
/* O:type of order; P0,P1: id's of servers */
def dgr(G,V,O,P0,P1)
{
  P = [P0,P1]; /* server list */
  map(ox_reset,P); /* reset servers */
  /* P0 executes non-homogenized computation */
  ox_cmo_rpc(P0,"dp_gr_main",G,V,0,1,O);
  /* P1 executes homogenized computation */
  ox_cmo_rpc(P1,"dp_gr_main",G,V,1,1,O);
  map(ox_push_cmd,P,262); /* 262 = OX_popCMO */
  F = ox_select(P); /* wait for data */
  /* F[0] is a server's id which is ready */
  R = ox_get(F[0]);
  if ( F[0] == P0 ) {
    Win = "nonhomo"; Lose = P1;
  } else {
    Win = "homo"; Lose = P0;
  }
  ox_reset(Lose); /* reset the loser */
  return [Win,R];
}
\end{verbatim}

% $OpenXM$
\section{Applications}

\subsection{Heterogeneous Servers}

\def\pd#1{ \partial_{#1} }

By using OpenXM, we can treat OpenXM servers essentially 
like a subroutine.
Since OpenXM provides a universal stack machine which does not
depend each servers, 
it is relatively easy to install new servers.
We can build a new computer math system by assembling
different OpenXM servers.
It is similar to building a toy house by LEGO blocks.

We will see two examples of custom-made systems
built by OpenXM servers.

\subsubsection{Computation of annihilating ideals by kan/sm1 and ox\_asir}

Let $D = {\bf Q} \langle x_1, \ldots, x_n , \pd{1}, \ldots, \pd{n} \rangle$
be the ring of differential operators.
For a given polynomial
$ f \in {\bf Q}[x_1, \ldots, x_n] $,
the annihilating ideal of $f^{-1}$ is defined as
$$ {\rm Ann}\, f^{-1} = \{ \ell \in D \,|\,
  \ell \bullet f^{-1} = 0 \}.
$$
Here, $\bullet$ denotes the action of $D$ to functions.
The annihilating ideal can be regarded as the maximal differential
equations for the function $f^{-1}$.
An algorithm to determine generators of the annihilating ideal
was given by Oaku (see, e.g., \cite[5.3]{sst-book}).
His algorithm reduces the problem to computations of Gr\"obner bases
in $D$ and to find the minimal integral root of a polynomial.
This algorithm (the function {\tt annfs}) is implemented by
kan/sm1 \cite{kan}, for Gr\"obner basis computation in $D$, and
{\tt ox\_asir}, to factorize polynomials to find the integral
roots.
These two OpenXM compliant systems are integrated by
the OpenXM protocol.

For example, the following is a sm1 session to find the annihilating
ideal for $f = x^3 - y^2 z^2$.
\begin{verbatim}
sm1>[(x^3-y^2 z^2) (x,y,z)] annfs ::
Starting ox_asir server.
Byte order for control process is network byte order.
Byte order for engine process is network byte order.
[[-y*Dy+z*Dz, 2*x*Dx+3*y*Dy+6, -2*y*z^2*Dx-3*x^2*Dy, 
-2*y^2*z*Dx-3*x^2*Dz, -2*z^3*Dx*Dz-3*x^2*Dy^2-2*z^2*Dx], 
 [-1,-139968*s^7-1119744*s^6-3802464*s^5-7107264*s^4
     -7898796*s^3-5220720*s^2-1900500*s-294000]] 
\end{verbatim}
The last polynomial is factored as
$-12(s+1)(3s+5)(3s+4)(6s+5)(6s+7)$
and the minimal integral root is $-1$
as shown in the output.

Similarly, 
an algorithm to stratify singularity 
\cite{oaku-advance}
is implemented by
kan/sm1 \cite{kan}, for Gr\"obner basis computation in $D$, and
{\tt ox\_asir}, for primary ideal decompositions.

\subsubsection{A Course on Solving Algebraic Equations}

Risa/Asir \cite{asir} is a general computer algebra system
which can be used for Gr\"obner basis computations for zero dimensional ideal
with ${\bf Q}$ coefficients.
However, it is not good at graphical presentations and
numerical methods.
We integrated Risa/Asir, ox\_phc (based on PHC pack by Verschelde \cite{phc}
for the polyhedral homotopy method) and
ox\_gnuplot (GNUPLOT) servers
to teach a course on solving algebraic equations.
This course was presented with the text book \cite{CLO},
which discusses 
on the Gr\"obner basis method and the polyhedral homotopy method
to solve systems of algebraic equations.
We taught the course
with a unified environment
controlled by the Asir user language, which is similar to C.
The following is an Asir session to solve algebraic equations by calling
the PHC pack (Figure \ref{katsura} is the output of {\tt [292]}):
\begin{verbatim}
[287] phc(katsura(7));
The detailed output is in the file tmp.output.*
The answer is in the variable Phc.
0
[290] B=map(first,Phc)$
[291] gnuplot_plotDots([],0)$
[292] gnuplot_plotDots(B,0)$
\end{verbatim}

\begin{figure}[htbp]
\epsfxsize=8.5cm
\epsffile{katsura7.ps}
\caption{The first components of the solutions to the system of algebraic equations Katsura 7.}
\label{katsura}
\end{figure}






%$OpenXM: OpenXM/doc/ascm2001p/bib.tex,v 1.4 2001/06/20 02:39:25 noro Exp $

\begin{thebibliography}{X}
\bibitem{OpenMath}
The OpenMath Esprit Consortium 
(Caprotti, O. and Cohen, A.M. Editors),
The OpenMath Standard. D1.3.2a (Public) \\
{\small {\tt http://www.nag.co.uk/projects/OpenMath}},
February, 1999.
%\bibitem{CLO}
%Cox, D., Little, J.,  O'Shea,
%{\it Using Algebraic Geometry}, Springer, 1998.
\bibitem{GKW}
Gray, S., Kajler, N. and Wang, P. S.,
Design and Implementation of MP, a Protocol for Efficient
  Exchange of Mathematical Expressions,
{\sl Journal of Symbolic Computation}, 1996.
\bibitem{Macaulay2}
Grayson, D. and Stillman, M.,
Macaulay2, a software system for research in algebraic geometry,
 {\small {\tt http://www.math.uiuc.edu/Macaulay2}}.
\bibitem{Singular}
Greuel, G.-M. et al., SINGULAR : a computer algebra system for polynomial
computations,
{\small {\tt http://www.mathematik.uni-kl.de/\~\,zca/Singular/}}.
\bibitem{tigers}
Hubert, B. and Thomas, R.,
{\tt TiGERS} --- computing Gr\"obner fans of toric
ideals, 1998.
{\small
{\tt http://www.math.tamu.edu/\~\,rekha/programs.html}}
\bibitem{omei}
Liao, W., Lin, D., and Wang, P.S.,
{OMEI: An Open Mathematical Engine Interface},
preprint.
\bibitem{gap}
Linton, S. and Solomon, A.,
OpenMath, IAMC and {\tt GAP},
preprint, 1999.
\bibitem{MPI} Message Passing Interface,
{\small {\tt http://www.mpi-forum.org}} 
\bibitem{netsolve}
NetSolve, {\small {\tt http://www.cs.utk.edu/netsolve}}
\bibitem{asir} 
Noro, M. et al., 
A Computer Algebra System {\tt Risa/Asir},  1993, 1995, 2000,
{\small {\tt ftp://archives.cs.ehime-u.ac.jp/pub/asir2000/}}
\bibitem{noro-mckay}
Noro, M. and McKay, J.,
Computation of replicable functions on Risa/Asir.
Proceedings of the Second International Symposium on
Symbolic Computation PASCO'97, ACM Press, 130-138 (1997).
\bibitem{ox-rfc-100}
Noro, M and Takayama, N., Design and Implementation
of OpenXM Client-Server Model and Common Mathematical Object Format
(OpenXM-RFC 100), 1996--2001.
%%
%\bibitem{oaku-advance}
%Oaku, T.,
%Algorithms for $b$-functions, restrictions, and algebraic local cohomology
%groups of $D$-modules.
%Advances in Applied Mathematics, {\bf 61}, 61--105, 1997.
%%
\bibitem{ox-rfc-101}
Ohara, K.,
Protocol to Start Engines (OpenXM-RFC 101),
2000.
\bibitem{openxm-web}
{\small {\tt http://www.math.kobe-u.ac.jp/OpenXM}}
or 
{\small {\tt http://www.openxm.org}}
\bibitem{openxm-1077}
{\small {\tt http://www/OpenXM/1.1.3/html/OpenXM-poster/func1/index.html}}
\bibitem{sst-book}
Saito, M., Sturmfels, B. and Takayama, N.,
{\it Gr\"obner Deformations of Hypergeometric Differential Equations}.
Algorithms and Computation in Mathematics {\bf 6}. Springer, 1999.
\bibitem{schefstrom}
Schefstr\"om, D.,
Building a highly integrated development environment using
preexisting parts.
In IFIP 11th World Computer Congress, San Francisco, USA.
\bibitem{Shoup}
Shoup, V., 
A new polynomial factorization algorithm and 
its implementation,
{\sl Journal of Symbolic Computation}, 20, 364-397, 1996.
\bibitem{kan}
	Takayama, N.,
	{\em Kan: A system for computation in
	algebraic analysis,} 1991 version 1,
        1994 version 2, the latest version is 2.991106. 
	{\tt \small ftp.math.kobe-u.ac.jp/pub/kan}
\bibitem{phc}
Verschelde, J.,
PHCpack: A general-purpose solver for polynomial systems by
homotopy continuation.  ACM Transaction on Mathematical Softwares, 25(2) 
251-276, 1999.
\bibitem{iamc}
Wang, P.,
Design and Protocol for Internet Accessible Mathematical Computation.
Technical Report ICM-199901-001, ICM/Kent State University, 1999.
\bibitem{mathlink}
Wolfram, S.,
{\it The Mathematica Book, Fourth Edition}.
1999, Cambridge University Press.
\end{thebibliography}

\end{document}
\endinput
%%


%%Text may be set as \emph{emph}.\\
%%Text may be set as \texttt{texttt}.\\
%%Text may be set as \underline{unterline}.\\
%%Text may be set as \textbf{textbf}.\\
%%Text may be set as \textrm{textrm}.\\
%%Text may be set as {\tiny tiny}.\\

%%\begin{figure}
%%\hrule
%%Nice Postscript, isn't it?
%%\begin{center}
%%\IfFileExists{graphicx.sty}{
%%  \includegraphics{body.eps}
%%}{
%%  Sorry, package \texttt{graphicx} not present.
%%}
%%\end{center}
%%Same, a little bit smaller:
%%\begin{center}
%%\IfFileExists{graphicx.sty}{
%%  \includegraphics[scale=.5]{body.eps}
%%  }{
%%  Sorry, package \texttt{graphicx} not present.
%%}
%%\end{center}
%%\caption{\label{fig-1}This is a nice floating figure}
%%\hrule
%%\end{figure}




