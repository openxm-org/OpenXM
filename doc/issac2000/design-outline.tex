% $OpenXM: OpenXM/doc/issac2000/design-outline.tex,v 1.2 2000/01/02 07:32:11 takayama Exp $

\section{Design Outline} 

As Schefstr\"om clarified in \cite{schefstrom},
integration of tools and softwares has three dimensions:
data, control, and user interface.

Data integration concerns with the exchange of data between different
softwares or same softwares.
OpenMath \cite{OpenMath} and MP (Multi Protocol) \cite{GKW} are,
for example, general purpose mathematical data protocols.
They provide a standard way to express mathematical objects.
For example,
\begin{verbatim}
 <OMOBJ>  <OMI> 123 </OMI> </OMOBJ>
\end{verbatim}
means the (OpenMath) integer $123$ in OpenMath/XML expression.

Control integration concerns with the establishment and management of
inter-software communications.
Control involves, for example, a way to ask computation to other processes
and a method to interrupt computations on servers from a client.
RPC, HTTP, MPI, PVM are regarded as a general purpose control protocols or
infrastructures.
MCP (Mathematical Communication Protocol)
by Wang \cite{iamc} is such a protocol specialized to mathematics.

Although, data and control are orthogonal to each other,
real world requires both.
NetSolv \cite{netsolve}, OpenMath$+$MCP, MP$+$MCP \cite{iamc},
and MathLink of Mathematica provide both data and control integration.
These are currently studied ways of data and control integration.
Each integration method has their own special features due to their
own design goals and design motivations.
OpenXM is a project aiming to integrate data, control and user interfaces
from a different emphasis of a set of design goals with other projects.
To explain our design outline, we start with a list of
our motivations.
\begin{enumerate}
\item Noro,  who is one of the authors of OpenXM, has developed a general
purpose computer algebra system Risa/Asir \cite{asir}.
A set of functions for interative distributed computations were introduced
in Risa/Asir version 950831 released in 1995.
The model of computation was RPC (remote procedure call)
and it had its own serialization method for objects.
A robust interruption method was provided by having two communication channels
like ftp, which implements the simple network management protocol.
As an application of this robust and interractive distributed computation
system, 
a huge Gr\"obner basis was computed
to determine all replicable functions by Noro and McKay \cite{noro-mckay}.
However, the protocol was closed in asir and we thought that we should
design an open protocol.
\item Takayama, who is also one of the authors of OpenXM, has developed
a special purpose computer algebra system Kan/sm1 \cite{kan},
which is a Gr\"obner engine for the ring of differential operators $D$ and
a package for computational algebraic geometry via D-module computations.
In order to implement algorithms in D-modules due to Oaku 
(see, e.g., \cite{sst-book}),
factorizations and primary ideal decompositions were necessary.
Kan/sm1 does not have an implementation for these and called
Risa/asir as a C library or a unix external program.
This approach was not satisfactory.
Especially, we could not write a clean interface code between these
two systems.
We thought that it is necessary to provide a data and control protocol
for Risa/asir to work as a server of factorization and primary ideal
decomposition.
\item The number of mathematical softwares is increasing rapidly in the last
decade of the 20th century.
These are usually ``expert'' systems for one area of mathematics
such as ideals, groups, numbers, polytopes, and so on.
They have their own interfaces and data formats.
Interfaces are usually specialied to a specific field of mathematics
or poor because developers do not have time for designing user interface
languages.
It is fine for intensive and serious users of these systems.
%% x2 stands for x^2, specialized for polynomial ring.
However, for users who want to explore a new area of mathematics with these
softwares or users who need these systems only occasionally,
a unified system will be more convinient.
For example, if we can call and use mathematical softwares
like CoCoa, GAP, Macaulay2, Porta, Singular, Snapea, $\ldots$
from Asir, Axion, Maple, muPAD, Mathematica, and so on,
it will be wonderful in research and education
of mathematics. This is an unification of user interfaces of mathematical
softwares.
\item  We believe that open integrated systems is a future of mathematical
softwares.
However, it might be just a dream without realizability.
We wanted to build a prototype system of such an open system by using
existing standards, technologies and several mathematical softwares.
We want to see how far we can go with this approach.
\end{enumerate}

Motivated with these, we started the OpenXM project with the following
fundamental architecture.
\begin{enumerate}
\item Communication is an exchange of messages. The messages are classifed into
three types:
DATA, COMMAND, and others.
The messages are called OX (OpenXM) messages.
Mathematical data are wrapped with {\it OX messages}.
We use standards of mathematical data formats such as OpenMath and MP
and our own data format ({\it CMO --- Common Mathematical Object format})
as data expressions.
\item Servers, which provide services to other processes, are stackmachines.
The stackmachine is called the
{\it OX stackmachine}.
Existing mathematical softwares are wrapped with this stackmachine.
Minimal requirements for a target software wrapped with the OX stackmachine
are as follows:
\begin{enumerate}
\item The target must have a seriealized interface such as a character based
interface.
\item An output of the target must be understandable for computer programs;
it should follow a grammer that can be parsed with other softwares.
\end{enumerate}
\end{enumerate}
We are implementing a package, OpenXM package,  
which aims to realize our wishes stated as motivations.
It is based on above fundamental architecture.
For example, the following is a command sequence to ask $1+1$ from
the asir client to the OX sm1 server:
\begin{verbatim}
  P = sm1_start();
  ox_push_cmo(P,1); ox_push_cmo(P,1);
  ox_execute_string(P,"add"); ox_pop_cmo(P);
\end{verbatim}
The current system, OpenXM on TCP/IP, 
uses client-server model and the TCP/IP is used for interprocess
communications.
A prototype OpenXM system on MPI \cite{MPI} already exists for Risa/asir and
a general OpenXM on MPI is a work in progress.
However, we focus only on the system based on TCP/IP in this paper.



