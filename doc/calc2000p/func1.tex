% $OpenXM: OpenXM/doc/calc2000p/func1.tex,v 1.3 2000/07/31 07:26:12 noro Exp $
\documentclass[twocolumn]{article}
\pagestyle{empty}
%\markright{ {\tt http://www.openxm.org} }
\usepackage{color}
\usepackage{epsfig}
\title{\huge \color{blue} 1077 functions are available
on our servers and libraries}
\author{} \date{}
\begin{document}
\maketitle

\noindent
\fbox{\huge {\color{green}Operations on Integers}}

\noindent
{\color{red} idiv},{\color{red} irem} (division with remainder),
{\color{red} ishift} (bit shifting),
{\color{red} iand},{\color{red} ior},{\color{red} ixor} (logical operations),
{\color{red} igcd},(GCD by various methods such as Euclid's algorithm and
the accelerated GCD algorithm),
{\color{red} fac} (factorial),
{\color{red} inv} (inverse modulo an integer),
{\color{red} random} (random number generator by the Mersenne twister algorithm).



\medbreak

\noindent
\fbox{\huge {\color{green}Ground Fields}}

\noindent
Arithmetics on various fields: the rationals, 
${\bf Q}(\alpha_1,\alpha_2,\ldots,\alpha_n)$
($\alpha_i$ is algebraic over ${\bf Q}(\alpha_1,\ldots,\alpha_{i-1})$),
$GF(p)$ ($p$ is a prime of arbitrary size), $GF(2^n)$.

\medbreak

\noindent
\fbox{\huge {\color{green}Operations on Polynomials}}

\noindent
{\color{red} sdiv }, {\color{red} srem } (division with remainder),
{\color{red} ptozp } (removal of the integer content),
{\color{red} diff } (differentiation),
{\color{red} gcd } (GCD over the rationals),
{\color{red} res } (resultant),
{\color{red} subst } (substitution),
{\color{red} umul} (fast multiplication of dense univariate polynomials 
by a hybrid method with Karatsuba and FFT+Chinese remainder),
{\color{red} urembymul\_precomp} (fast dense univariate polynomial 
division with remainder by the fast multiplication and 
the precomputed inverse of a divisor),

\noindent
\fbox{\huge {\color{green}Polynomial Factorization}}
{\color{red} fctr } (factorization over the rationals),
{\color{red} fctr\_ff } (univariate factorization over finite fields),
{\color{red} af } (univariate factorization over algebraic number fields),
{\color{red} sp} (splitting field computation).

\medbreak

\noindent
\fbox{\huge {\color{green} Groebner basis}} 

\noindent
{\color{red} dp\_gr\_main } (Groebner basis computation of a polynomial ideal 
over the rationals by the trace lifting),
{\color{red} dp\_gr\_mod\_main } (Groebner basis over small finite fields),
{\color{red} tolex } (Modular change of ordering for a zero-dimensional ideal),
{\color{red} tolex\_gsl } (Modular rational univariate representation 
for a zero-dimensional ideal),
{\color{red} dp\_f4\_main } ($F_4$ over the rationals),
{\color{red} dp\_f4\_mod\_main } ($F_4$ over small finite fields).

\medbreak
\noindent
\fbox{\huge {\color{green} Ideal Decomposition}} 

\noindent
{\color{red} primedec} (Prime decomposition of the radical),
{\color{red} primadec} (Primary decomposition of ideals by Shimoyama/Yokoyama algorithm).

\medbreak

\noindent
\fbox{\huge {\color{green} Quantifier Elimination}} 

\noindent
{\color{red} qe} (real quantifier elimination in a linear and 
quadratic first-order formula),
{\color{red} simpl} (heuristic simplification of a first-order formula).

{\scriptsize
\begin{verbatim}
[0] MTP2 = ex([x11,x12,x13,x21,x22,x23,x31,x32,x33],
x11+x12+x13 @== a1 @&& x21+x22+x23 @== a2 @&& x31+x32+x33 @== a3 
@&& x11+x21+x31 @== b1 @&& x12+x22+x32 @== b2 @&& x13+x23+x33 @== b3
@&& 0 @<= x11 @&& 0 @<= x12 @&& 0 @<= x13 @&& 0 @<= x21
@&& 0 @<= x22 @&& 0 @<= x23 @&& 0 @<= x31 @&& 0 @<= x32 @&& 0 @<= x33)$
[1] TSOL= a1+a2+a3@=b1+b2+b3 @&& a1@>=0 @&& a2@>=0 @&& a3@>=0
@&& b1@>=0 @&& b2@>=0 @&& b3@>=0$
[2] QE_MTP2 = qe(MTP2)$
[3] qe(all([a1,a2,a3,b1,b2,b3],QE_MTP2 @equiv TSOL));
@true
\end{verbatim}}
\medbreak

\noindent
\fbox{\huge {\color{green} Visualization of curves}} 

\noindent
{\color{red} plot} (plotting of a univariate function),
{\color{red} ifplot} (plotting zeros of a bivariate polynomial),
{\color{red} conplot} (contour plotting of a bivariate polynomial function).

\medbreak

\noindent
\fbox{\huge {\color{green} Miscellaneous functions}} 

\noindent
{\color{red} det} (determinant),
{\color{red} qsort} (sorting of an array by the quick sort algorithm),
{\color{red} eval} (evaluation of a formula containing transcendental functions
such as 
{\color{red} sin}, {\color{red} cos}, {\color{red} tan}, {\color{red} exp},
{\color{red} log})
{\color{red} roots} (finding all roots of a univariate polynomial),
{\color{red} lll} (computation of an LLL-reduced basis of a lattice).

\medbreak
\vfill
\noindent
\rightline{ {\color{red} {\tt http://www.openxm.org} }}

\end{document}
