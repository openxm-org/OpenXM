%$OpenXM$
\documentclass[a4paper,12pt]{article}
\usepackage{amsmath,amssymb}
\usepackage{graphicx,psfrag}

\title{The usage of ``dualgraph" by examples}
\begin{document}
\maketitle

\begin{enumerate}
\item Download Risa/Asir from {\tt http://www.math.kobe-u.ac.jp/Asir/}\\
\item Install Risa/Asir according to manual.\\
\item Carry out Asir.
\end{enumerate}

The package is the file {\tt dualgraph.rr} in the directory 'asir-contrib/fj\_curve'.

Attention: The file ``{\tt dualgraph.rr}" contains Japanese jis code.

\begin{verbatim}
This is Risa/Asir, Version 20051106 (Kobe Distribution).
Copyright (C) 1994-2000, all rights reserved, FUJITSU LABORATORIES LIMITED.
Copyright 2000-2005, Risa/Asir committers, http://www.openxm.org/.
GC 6.5 Copyright 1988-2005, H-J. Boehm, A. J. Demers, Xerox, SGI, HP.
PARI 2.0.17, copyright 1989-1999, C. Batut, K. Belabas, D. Bernardi,
   H. Cohen and M. Olivier.
Debug windows of ox servers will not be opened. Set Xm_noX=0 to open it.
OpenXM/Risa/Asir-Contrib(20040302), Copyright 2000-2004, OpenXM.org committers
ox_help(0); ox_help("keyword"); ox_grep("keyword");
     for help messages (unix version only).
http://www.math.kobe-u.ac.jp/OpenXM/Current/doc/index-doc.html
[1217] load("gr")$
[1323] load("sp")$
[1425] load("fj_curve/dualgraph.rr")$
[1479] dual_graph((y^2-x^3)^2-y^7)$
*** Newton Polygon ***
[[0,4],[6,0]]
[[0,2],[9,0]]
***** Dual Graph *****
[3,[2,b1],2]
b1:[2,2,2,3,[1,b2],2]
b2:*
[1480]
\end{verbatim}

This result means following dual graph.

 \psfrag{-1}{$-1$}
 \psfrag{-2}{$-2$}
 \psfrag{-3}{$-3$}
 \psfrag{-5}{$-5$}
 \psfrag{*}{$*$}
 \centerline{
  \includegraphics*[height=5cm]{g-dual1.eps}
 }

\begin{verbatim}
[194] F=((x^2-y^5)^2-x^5)*((-x^2+y^3)^2-y^9)
*((x^2+y^3)^2-x^7)*(x^4+y^5)$
[195] dual_graph(F)$
*** Newton Polygon ***
[[0,27],[4,17],[12,5],[16,0]]
[[0,2],[5,0]]
[[0,2],[6,0]]
[[0,2],[9,0]]
***** Dual graph *****
[2,[2,b1],4,[4,b2,b3],2,[1,b4],5]
b1:[2,3,[1,b5],2]
b5:*
b2:[2,2,[1,b6,b7]]
b6:*
b7:*
b3:[2,2,2,3,[1,b8],2]
b8:*
b4:*
\end{verbatim}

``b1,b2,b3,b4" mean another branches from these verticies.

\begin{center}
{\large [2,[2,b1],4,[4,b2,b3],2,[1,b4],5]}
\end{center}
$$\big\Updownarrow$$
 \psfrag{b1}{b1}
 \psfrag{b2}{b2}
 \psfrag{b3}{b3}
 \psfrag{b4}{b4}
 \centerline{
  \includegraphics*[height=3.5cm]{ex2dual1.eps}
 }
\vspace*{6pt}

The list "b1:..." means the dual graph of the branch "b1".

\begin{center}
{\large b1:[2,3,[1,b5],2]}
\end{center}
$$\big\Updownarrow$$
 \psfrag{b1}{b1}
 \psfrag{b5}{b5}
 \centerline{
  \includegraphics*[height=4cm]{ex2dual2.eps}
 }

Therefor we get following dual graph.

 \psfrag{1}{$-1$}
 \psfrag{2}{$-2$}
 \psfrag{3}{$-3$}
 \psfrag{4}{$-4$}
 \psfrag{5}{$-5$}
 \centerline{
  \includegraphics*[width=11cm]{dualex2.eps}
 }

\end{document}
